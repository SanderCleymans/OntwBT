\documentclass{article}
\usepackage[dutch]{babel}
\usepackage{graphicx}
\author{Sander Cleymans}
\date{\today}
\title{Ontwikkeling van Bedrijfstoepassingen \\
		Casestudie Ryanair}

\begin{document} 

\maketitle

\newpage  

\tableofcontents

\newpage
\part{Samenvatting voor de raad van Bestuur}

In dit verslag maak ik casestudie voor Ryanair in 2014.
Allereerst worden de algemene kenmerken van het bedrijf aangehaald, onder meer de concurrentie en de standpunten van Ryanair tegenover bepaalde 'vaste waarden' uit de industrie.
   
De concurrentie van Ryanair is door de jaren enkele keren verandert, maar een van de vaste waarden bij de Europese Low-Cost Carriers (LCC's) en een geduchte concurrent van Ryanair, is het Britse easyJet. Daarnaast is er natuurlijk nog de grote groep traditionele luchtvaartmaatschappijen die allemaal hun eigen doelpubliek hebben, maar allemaal vechten om de overgebleven klanten te kunnen overtuigen om bij hen te vliegen.
	
In het tweede stuk van het verslag ga ik iets dieper in op alle verschillende kritieke succesfactoren (CSF's) gerelateerd aan Ryanair:

\begin{itemize}
\item Trouw blijven aan de de point-to-point verbindingen die ze aanhouden met de secundaire(!) luchthavens.
\item De laagste prijs behouden.
\item Zoeken naar innovatieve oplossingen zodat de prijzen laag blijven.
\item Vooruitstrevende beslissingen durven nemen, enerzijds voor publiciteit, maar ook ter bevordering van prijs/kwaliteit en klantgerichtheid.
\end{itemize}

Bij elk van deze CSF's wordt er uitgelegd van waar deze komt en er worden er nog enkele andere aangehaald.
Hieruit is er een clustering analyse gedaan, en 

\begin{center}
\begin{Huge}
TODO
\end{Huge}
\end{center} 

\newpage
\part{Concurrentie en Competitie}

De huidige toestand van de luchtvaart-markt, laat ons toe om redelijk mooi in te schatten hoe sterk elke maatschappij zich heeft kunnen vestigen op een bepaalde plaats tegenover zijn onmiddellijke concurrenten.
Er is een duidelijk onderscheid te maken tussen de directe concurrenten en de indirecte concurrentie.

\section{Ryanair en easyJet}

De geschiedenis en de concurrentie tussen deze twee maatschappijen gaat terug tot het begin van de openstelling van het Europese luchtvaartindustrie in 1997. Momenteel zijn het nummer 1 en 2 op de Europese markt, maar daar is een heftige concurrentiestrijd aan vooraf gegaan. Omdat beide maatschappijen zich op dezelfde doelgroep richten (goedkope, point-to-point vluchten), trachten ze steeds de andere af te troeven op een bepaald vlak. Zij het om de prijzen voor een bepaalde route naar beneden te houden, zij het om het snelst uit te breiden naar nieuwe landen,...

Beide landen houden er dezelfde strategie op na: een no-frills, point-to-point vlucht aanbieden voor de laagst mogelijke prijs, zonder er zelf aan ten onder te gaan. Ze doen dit beide een zeer strak businessplan aan te houden: minimaal aantal verschillende vliegtuigtypes (bij Ryanair momenteel zelfs beperkt tot een type, de Boeing 737-800), alle extra's worden ook extra aangerekend en een beperkt aantal routes.

Een essentieel verschil tussen de twee is echter dat Ryanair zich strikt beperkt tot secundaire luchthavens, terwijl easyJet wel de grote aandoet: Paris Charles de Gaulle, London Gatwick, Brussels Airport en zowat elke andere grote Europese stad. Daar tegenover staat dat Ryanair (meestal) kleinere luchthavens aandoet: Brussel-Charleroi, Frankfurt-Hahn,... De laatste paar jaren zijn er hier echter enkele grotere bij gekomen. Dit omdat het zijn doelpubliek wil uitbreiden en in de running wil blijven met zijn grootste rivaal easyJet.\footnote{http://news.airwise.com/story/view/1385577015.html} Hierdoor wijkt het wel af van zijn eigen businessplan, maar met een goede verantwoording.

De strijd tussen de 2 heeft er voor gezorgd dat beide met enorm lage prijzen vluchten aanbieden naar zowat gans Europa (en enkele bestemmingen daarbuiten, maar voor dit verslag zullen we het bij Europa houden). Door hun geslaagde strategie zijn Ryanair en easyJet momenteel de twee grootste LCC's van Europa, met respectievelijk 81.4 en 59.2 miljoen passagiers in 2013.

\section{Andere concurrenten}

Hier komen we aan bij de andere concurrenten van Ryanair, waaronder we ondermeer de vele kleine (veelal lokale) maatschappijen terugvinden, maar ook andere grote spelers op de markt: British Airways, Lufthansa, Iberia,... 
Het grootste verschil met Ryanair en easyJet is hier dat deze maatschappijen niet specifiek naar een low-budget model willen gaan. Zo worden, in tegenstelling met Ryanair (e.a.), zogenaamde 'ancillary services' niet extra aangerekend. Er wordt ook gevlogen naar grote luchthavens, men maakt gebruik van persoonlijk contact bij in/uitchecken van een vlucht en de boeking van een ticket, waardat bij Ryanair zo goed als alles hiervan elektronisch gebeurd. Verder maken ze gebruik van een uitgebreidere vloot, wat hun iets meer vrijheid geeft over bezetting van vluchten en routes.
De routes van deze maatschappijen zijn ook van een andere soort, ze vertrekken meestal alleen uit hun thuisland, naar bepaalde bestemmingen (en terug natuurlijk). Dit in tegenstelling met Ryanair die bijvoorbeeld ook vluchten van Belgi\"e naar Frankrijk voorziet.

\part{Critical Success Factors van Ryan air}








\end{document}