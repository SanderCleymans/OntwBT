\documentclass{article}
\usepackage[dutch]{babel}
\author{Sander Cleymans}
\date{\today}
\title{Ontwikkeling van Bedrijfstoepassingen \\
		Casestudie Ryanair}


\begin{document} 
\maketitle
\newpage  

\tableofcontents



\newpage
\part{Samenvatting voor de raad van Bestuur}

In dit verslag maak ik casestudie voor Ryanair in 2014.
Allereerst worden de algemene kenmerken van het bedrijf aangehaald, onder meer de concurrentie en de standpunten van Ryanair tegenover bepaalde 'vaste waarden' uit de industrie.
   
De concurrentie van Ryanair is door de jaren enkele keren verandert, maar een van de vaste waarden bij de Europese Low-Cost Carriers (LCC's) en een geduchte concurrent van Ryanair, is het Britse easyJet. Daarnaast is er natuurlijk nog de grote groep traditionele luchtvaartmaatschappijen die allemaal hun eigen doelpubliek hebben, maar allemaal vechten om de overgebleven klanten te kunnen overtuigen om bij hen te vliegen.
	
In het tweede stuk van het verslag ga ik iets dieper in op alle verschillende kritieke succesfactoren (CSF's) gerelateerd aan Ryanair:
\begin{itemize}
\item Trouw blijven aan de de point-to-point verbindingen die ze aanhouden met de secundaire(!) luchthavens.
\item De laagste prijs behouden.
\item Zoeken naar innovatieve oplossingen zodat de prijzen laag blijven.
\item Vooruitstrevende beslissingen durven nemen, enerzijds voor publiciteit, maar ook ter bevordering van prijs/kwaliteit en klantgerichtheid.
\end{itemize}

Bij elk van deze CSF's wordt er uitgelegd van waar deze komt en er worden er nog enkele andere aangehaald.
Hieruit is er een clustering analyse gedaan, en 

TODO
\newpage
\part{Concurrentie en Competitie}

De huidige toestand van de luchtvaart-markt, laat ons toe om redelijk mooi in te schatten hoe sterk elke maatschappij zich heeft kunnen vestigen op een bepaalde plaats tegenover zijn onmiddellijke concurrenten.
Er is een duidelijk onderscheid te maken tussen de directe concurrenten en de indirecte concurrentie.

\section{Ryanair en easyJet}

De geschiedenis en de concurrentie van deze 2 maatschappijen gaat terug tot het begin van de openstelling van het Europese luchtvaartindustrie in 1997. Momenteel zijn het nummer 1 en 2 op de Europese markt, maar daar is een heftige concurrentiestrijd aan vooraf gegaan. Omdat beide maatschappijen zich op dezelfde doelgroep richten (goedkope, point-to-point vluchten), trachten ze steeds de andere af te troeven op een bepaald vlak. Zij het om de prijzen voor een bepaalde route naar beneden te houden, zij het om het snelst uit te breiden naar nieuwe landen,...







\end{document}